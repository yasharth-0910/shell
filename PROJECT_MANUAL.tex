% Custom Bash Shell Environment - Complete Project Report and User Manual
% Author: Senior Developer
% Date: October 2025

\documentclass[12pt,a4paper]{report}

% Packages
\usepackage[utf8]{inputenc}
\usepackage[english]{babel}
\usepackage{geometry}
\usepackage{graphicx}
\usepackage{listings}
\usepackage{xcolor}
\usepackage{hyperref}
\usepackage{fancyhdr}
\usepackage{titlesec}
\usepackage{tocloft}
\usepackage{enumitem}
\usepackage{booktabs}
\usepackage{longtable}
\usepackage{array}
\usepackage{multirow}
\usepackage{tcolorbox}
\usepackage{fontawesome5}

% Page setup
\geometry{
    a4paper,
    left=2.5cm,
    right=2.5cm,
    top=2.5cm,
    bottom=2.5cm
}

% Colors
\definecolor{codebackground}{RGB}{245,245,245}
\definecolor{codecomment}{RGB}{0,128,0}
\definecolor{codekeyword}{RGB}{0,0,255}
\definecolor{codestring}{RGB}{163,21,21}
\definecolor{shellcolor}{RGB}{52,101,164}
\definecolor{titlecolor}{RGB}{0,102,204}

% Code listing setup
\lstset{
    language=bash,
    backgroundcolor=\color{codebackground},
    basicstyle=\ttfamily\small,
    breaklines=true,
    captionpos=b,
    commentstyle=\color{codecomment},
    keywordstyle=\color{codekeyword}\bfseries,
    stringstyle=\color{codestring},
    numbers=left,
    numberstyle=\tiny\color{gray},
    frame=single,
    rulecolor=\color{black},
    showstringspaces=false,
    tabsize=4
}

% Hyperref setup
\hypersetup{
    colorlinks=true,
    linkcolor=blue,
    filecolor=magenta,
    urlcolor=cyan,
    pdftitle={Custom Bash Shell Environment - Complete Guide},
    pdfauthor={Senior Developer},
    pdfsubject={Shell Scripting},
    pdfkeywords={Bash, Shell, Linux, System Administration}
}

% Headers and footers
\pagestyle{fancy}
\fancyhf{}
\fancyhead[L]{\leftmark}
\fancyhead[R]{Custom Bash Shell Environment}
\fancyfoot[C]{\thepage}

% Title formatting
\titleformat{\chapter}[display]
{\normalfont\huge\bfseries\color{titlecolor}}{\chaptertitlename\ \thechapter}{20pt}{\Huge}

% Document begins
\begin{document}

% Title page
\begin{titlepage}
    \centering
    \vspace*{2cm}
    
    {\Huge\bfseries Custom Bash Shell Environment\\[0.5cm]}
    {\Large Complete Project Report \& User Manual\\[1.5cm]}
    
    \rule{\textwidth}{1.5pt}\\[0.5cm]
    
    {\Large\bfseries A Professional Shell Scripting Suite\\[0.3cm]}
    {\large Version 1.0.0\\[2cm]}
    
    \includegraphics[width=0.3\textwidth]{placeholder_logo}
    % Note: Replace with actual logo if available
    \\[2cm]
    
    {\Large\textbf{Author:} Senior Developer\\[0.5cm]}
    {\large October 2025\\[1cm]}
    
    \vfill
    
    {\large\textit{A comprehensive collection of 22 utility scripts\\
    for system administration, networking, and productivity}}
    
    \vspace{1cm}
    
    \rule{\textwidth}{1.5pt}
\end{titlepage}

% Abstract
\chapter*{Abstract}
\addcontentsline{toc}{chapter}{Abstract}

This document serves as a comprehensive guide to the Custom Bash Shell Environment project, a professional-grade shell scripting suite developed for Linux/Unix systems. The project encompasses a fully-functional custom shell interface along with 22 specialized utility scripts categorized into five distinct domains: System Administration, Network Services, Mathematical Operations, Interactive Utilities, and Miscellaneous Tools.

The custom shell (MyShell v1.0.0) provides an intuitive command-line interface with built-in commands, command history support, and seamless integration with all utility scripts. Each script is designed following industry best practices, featuring comprehensive error handling, input validation, and user-friendly output formatting.

This manual provides detailed information about project architecture, installation procedures, configuration options, and extensive usage examples for each script. It is intended for system administrators, developers, students, and anyone interested in shell scripting and Linux system automation.

\textbf{Keywords:} Bash Scripting, Shell Programming, System Administration, Linux Utilities, Command-Line Interface, Network Programming, Automation

\tableofcontents
\listoffigures
\listoftables

% Chapter 1: Introduction
\chapter{Introduction}

\section{Project Overview}

The Custom Bash Shell Environment is a sophisticated shell scripting project that demonstrates advanced concepts in operating systems, process management, file systems, and network programming. Built entirely using Bash 4.0+, this project provides both educational value and practical utility for everyday system administration tasks.

\subsection{Project Motivation}

Modern system administrators and developers often need quick access to various system monitoring, networking, and utility tools. While many individual tools exist, they are often scattered across different locations and lack a unified interface. This project addresses this gap by providing:

\begin{itemize}[leftmargin=*]
    \item A centralized custom shell environment
    \item 22 ready-to-use utility scripts
    \item Consistent interface and error handling
    \item Comprehensive documentation
    \item Easy extensibility for custom scripts
\end{itemize}

\subsection{Key Features}

\begin{tcolorbox}[colback=blue!5!white,colframe=blue!75!black,title=\textbf{Core Features}]
\begin{itemize}[leftmargin=*]
    \item \textbf{Custom Shell Interface:} Interactive command-line environment with 7 built-in commands
    \item \textbf{22 Utility Scripts:} Covering system administration, networking, mathematics, and productivity
    \item \textbf{Modular Design:} Each script is independent and can be used standalone
    \item \textbf{Professional Quality:} Comprehensive error handling, input validation, and user feedback
    \item \textbf{Cross-Platform:} Compatible with major Linux distributions (Ubuntu, Fedora, Arch, etc.)
    \item \textbf{Extensible Architecture:} Easy to add new scripts and customize existing ones
\end{itemize}
\end{tcolorbox}

\section{Technology Stack}

\begin{table}[h]
\centering
\begin{tabular}{|l|p{10cm}|}
\hline
\textbf{Component} & \textbf{Technology/Tool} \\ \hline
Primary Language & Bash 4.0+ \\ \hline
Core Utilities & coreutils (ls, cat, grep, awk, sed, etc.) \\ \hline
Network Tools & netcat (nc), curl \\ \hline
Math Processing & bc (basic calculator) \\ \hline
JSON Parsing & jq (optional) \\ \hline
System Monitoring & lm-sensors (optional) \\ \hline
Platform & Linux/Unix systems \\ \hline
License & MIT License \\ \hline
\end{tabular}
\caption{Technology Stack Overview}
\label{tab:tech_stack}
\end{table}

\section{Project Statistics}

\begin{table}[h]
\centering
\begin{tabular}{|l|r|}
\hline
\textbf{Metric} & \textbf{Value} \\ \hline
Total Scripts & 22 \\ \hline
Lines of Code & $\sim$3,500+ \\ \hline
Documentation Files & 7 \\ \hline
Installation Scripts & 2 \\ \hline
Script Categories & 5 \\ \hline
Total Project Files & 32+ \\ \hline
Documentation Size & 63 KB \\ \hline
\end{tabular}
\caption{Project Statistics}
\label{tab:project_stats}
\end{table}

\section{Target Audience}

This project is designed for:

\begin{itemize}[leftmargin=*]
    \item \textbf{System Administrators:} Monitoring and managing Linux servers
    \item \textbf{DevOps Engineers:} Automation and deployment tasks
    \item \textbf{Developers:} Quick access to development utilities
    \item \textbf{Students:} Learning shell scripting and OS concepts
    \item \textbf{Linux Enthusiasts:} Power users seeking productivity tools
\end{itemize}

% Chapter 2: Installation and Setup
\chapter{Installation and Setup}

\section{System Requirements}

\subsection{Minimum Requirements}

\begin{table}[h]
\centering
\begin{tabular}{|l|l|}
\hline
\textbf{Component} & \textbf{Requirement} \\ \hline
Operating System & Linux/Unix (kernel 2.6+) \\ \hline
Bash Version & 4.0 or higher \\ \hline
Disk Space & 10 MB \\ \hline
RAM & 512 MB \\ \hline
\end{tabular}
\caption{Minimum System Requirements}
\label{tab:min_req}
\end{table}

\subsection{Recommended Requirements}

\begin{itemize}[leftmargin=*]
    \item \textbf{OS:} Ubuntu 20.04+, Fedora 30+, or Arch Linux
    \item \textbf{Bash:} Version 5.0+
    \item \textbf{RAM:} 1 GB or more
    \item \textbf{Network:} Internet connection (for weather and Reddit scripts)
\end{itemize}

\section{Prerequisites Installation}

\subsection{Ubuntu/Debian Systems}

\begin{lstlisting}[caption={Installing Prerequisites on Ubuntu/Debian}]
# Update package list
sudo apt-get update

# Install required packages
sudo apt-get install -y bash curl netcat bc coreutils

# Install optional packages
sudo apt-get install -y jq lm-sensors

# Configure sensors (if needed)
sudo sensors-detect
\end{lstlisting}

\subsection{Fedora/RHEL/CentOS Systems}

\begin{lstlisting}[caption={Installing Prerequisites on Fedora/RHEL}]
# Install required packages
sudo dnf install -y bash curl nc bc coreutils

# Install optional packages
sudo dnf install -y jq lm_sensors

# Configure sensors
sudo sensors-detect
\end{lstlisting}

\subsection{Arch Linux Systems}

\begin{lstlisting}[caption={Installing Prerequisites on Arch Linux}]
# Install required packages
sudo pacman -S bash curl gnu-netcat bc coreutils

# Install optional packages
sudo pacman -S jq lm_sensors
\end{lstlisting}

\section{Project Installation}

\subsection{Method 1: Automated Installation}

The recommended method is using the automated installation script:

\begin{lstlisting}[caption={Automated Installation Process}]
# Navigate to project directory
cd /path/to/custom-bash-shell

# Make installer executable
chmod +x install.sh

# Run the installer
./install.sh
\end{lstlisting}

The installation script performs the following tasks:

\begin{enumerate}[leftmargin=*]
    \item Checks for required and optional dependencies
    \item Makes all scripts executable
    \item Creates configuration directory
    \item Offers to add shell to system PATH
    \item Optionally creates desktop launcher
    \item Displays installation summary
\end{enumerate}

\subsection{Method 2: Manual Installation}

For manual installation:

\begin{lstlisting}[caption={Manual Installation Steps}]
# Make main shell executable
chmod +x myshell.sh

# Make all scripts executable
chmod +x scripts/*.sh

# Create config directory
mkdir -p config
touch config/.myshell_history

# Optional: Add to PATH
echo 'export PATH="$PATH:/path/to/custom-bash-shell"' >> ~/.bashrc
source ~/.bashrc
\end{lstlisting}

\section{Verification}

After installation, verify the setup:

\begin{lstlisting}[caption={Verifying Installation}]
# Check if shell is executable
./myshell.sh

# Inside the shell, verify scripts
list

# Test a simple script
Addition 10 20

# Exit shell
exit
\end{lstlisting}

Expected output for \texttt{Addition 10 20}:
\begin{lstlisting}[frame=none,numbers=none]
=== Addition Calculator ===

10 + 20 = 30
\end{lstlisting}

% Chapter 3: Custom Shell Overview
\chapter{Custom Shell (myshell.sh)}

\section{Shell Architecture}

The custom shell (MyShell v1.0.0) is implemented in \texttt{myshell.sh} and provides a complete command-line interface with the following components:

\subsection{Core Components}

\begin{enumerate}[leftmargin=*]
    \item \textbf{Configuration Management:} Handles shell settings, paths, and color schemes
    \item \textbf{Command Parser:} Tokenizes and interprets user input
    \item \textbf{Built-in Commands:} Seven essential commands for shell navigation
    \item \textbf{Script Executor:} Discovers and executes utility scripts
    \item \textbf{History Manager:} Stores and retrieves command history
    \item \textbf{Signal Handler:} Manages interrupt signals (SIGINT, SIGTERM)
\end{enumerate}

\section{Starting the Shell}

\begin{lstlisting}[caption={Launching the Custom Shell}]
# From project directory
./myshell.sh

# Or if added to PATH
myshell.sh
\end{lstlisting}

Upon starting, you will see the welcome banner:

\begin{tcolorbox}[colback=gray!10,colframe=gray!50,title=Welcome Banner]
\begin{verbatim}
╔════════════════════════════════════════════════════════════╗
║                                                            ║
║           MyShell v1.0.0 - Custom Shell Environment       ║
║                                                            ║
║  Type 'help' for available commands                       ║
║  Type 'list' to see all utility scripts                   ║
║  Type 'exit' to quit                                      ║
║                                                            ║
╚════════════════════════════════════════════════════════════╝
\end{verbatim}
\end{tcolorbox}

\section{Built-in Commands}

\subsection{help - Display Help Information}

\textbf{Syntax:} \texttt{help}

\textbf{Description:} Displays comprehensive help information about available commands and script categories.

\textbf{Example:}
\begin{lstlisting}[caption={Using the help command}]
MyShell:~$ help
\end{lstlisting}

\textbf{Output:}
\begin{itemize}[leftmargin=*]
    \item List of all built-in commands
    \item Script categories overview
    \item Usage instructions
\end{itemize}

\subsection{list - List All Available Scripts}

\textbf{Syntax:} \texttt{list}

\textbf{Description:} Displays all available utility scripts organized by category.

\textbf{Example:}
\begin{lstlisting}[caption={Listing all scripts}]
MyShell:~$ list
\end{lstlisting}

\textbf{Output Categories:}
\begin{enumerate}[leftmargin=*]
    \item System Administration scripts
    \item Network Services scripts
    \item Mathematical Operations scripts
    \item Interactive Utilities scripts
    \item Miscellaneous scripts
\end{enumerate}

\subsection{cd - Change Directory}

\textbf{Syntax:} \texttt{cd [directory]}

\textbf{Description:} Changes the current working directory.

\textbf{Examples:}
\begin{lstlisting}[caption={Using cd command}]
# Change to home directory
MyShell:~$ cd

# Change to specific directory
MyShell:~$ cd /var/log

# Change to parent directory
MyShell:/var/log$ cd ..
\end{lstlisting}

\subsection{pwd - Print Working Directory}

\textbf{Syntax:} \texttt{pwd}

\textbf{Description:} Displays the current working directory path.

\textbf{Example:}
\begin{lstlisting}[caption={Using pwd command}]
MyShell:~$ pwd
/home/username
\end{lstlisting}

\subsection{clear - Clear Screen}

\textbf{Syntax:} \texttt{clear}

\textbf{Description:} Clears the terminal screen and displays the welcome banner again.

\subsection{history - Show Command History}

\textbf{Syntax:} \texttt{history}

\textbf{Description:} Displays the last 20 commands executed in the shell.

\textbf{Example:}
\begin{lstlisting}[caption={Viewing command history}]
MyShell:~$ history
=== Command History ===
     1  help
     2  list
     3  Addition 10 20
     4  pwd
     5  history
\end{lstlisting}

\subsection{exit / quit - Exit the Shell}

\textbf{Syntax:} \texttt{exit} or \texttt{quit}

\textbf{Description:} Exits the custom shell gracefully.

\textbf{Example:}
\begin{lstlisting}[caption={Exiting the shell}]
MyShell:~$ exit
Goodbye from MyShell!
\end{lstlisting}

\section{Running Scripts}

Scripts can be executed with or without the \texttt{.sh} extension:

\begin{lstlisting}[caption={Different ways to run scripts}]
# With .sh extension
MyShell:~$ DirectorySize.sh /home/user

# Without .sh extension
MyShell:~$ DirectorySize /home/user

# With arguments
MyShell:~$ CPU 85

# Interactive scripts
MyShell:~$ Simplecalc
\end{lstlisting}

% Chapter 4: System Administration Scripts
\chapter{System Administration Scripts}

This chapter provides detailed documentation for all nine system administration scripts.

\section{DirectorySize.sh - Directory Size Analyzer}

\subsection{Purpose}
Analyzes and reports the size of directories, showing the total size and the top 10 largest subdirectories.

\subsection{Syntax}
\begin{lstlisting}[frame=single,numbers=none]
DirectorySize.sh [directory_path]
\end{lstlisting}

\subsection{Parameters}
\begin{itemize}[leftmargin=*]
    \item \textbf{directory\_path} (optional): Path to the directory to analyze. Defaults to current directory if not specified.
\end{itemize}

\subsection{Features}
\begin{itemize}[leftmargin=*]
    \item Displays total directory size in human-readable format
    \item Lists top 10 largest subdirectories sorted by size
    \item Color-coded output for better readability
    \item Error handling for non-existent directories
\end{itemize}

\subsection{Usage Examples}

\textbf{Example 1: Analyze current directory}
\begin{lstlisting}[caption={Analyzing current directory}]
MyShell:~$ DirectorySize
\end{lstlisting}

\textbf{Example 2: Analyze specific directory}
\begin{lstlisting}[caption={Analyzing specific directory}]
MyShell:~$ DirectorySize /var/log
\end{lstlisting}

\textbf{Example 3: Analyze home directory}
\begin{lstlisting}[caption={Analyzing home directory}]
MyShell:~$ DirectorySize ~
\end{lstlisting}

\subsection{Sample Output}
\begin{tcolorbox}[colback=gray!10,colframe=gray!50]
\begin{verbatim}
=== Directory Size Report ===

Analyzing: /var/log

Total Size:
1.2G    /var/log

Top 10 Largest Subdirectories:
450M    /var/log/journal
320M    /var/log/apache2
180M    /var/log/syslog
95M     /var/log/kern.log
...

Analysis complete!
\end{verbatim}
\end{tcolorbox}

\subsection{Use Cases}
\begin{enumerate}[leftmargin=*]
    \item Disk space management and cleanup
    \item Identifying large directories consuming storage
    \item Regular system auditing
    \item Before backup operations
\end{enumerate}

\subsection{Error Handling}
\begin{itemize}[leftmargin=*]
    \item Validates directory existence before analysis
    \item Displays error message for invalid paths
    \item Handles permission denied errors gracefully
\end{itemize}

\section{Test-File.sh - File and Directory Tester}

\subsection{Purpose}
Comprehensive file and directory testing utility that checks existence, type, permissions, and detailed metadata.

\subsection{Syntax}
\begin{lstlisting}[frame=single,numbers=none]
Test-File.sh <file_or_directory_path>
\end{lstlisting}

\subsection{Parameters}
\begin{itemize}[leftmargin=*]
    \item \textbf{file\_or\_directory\_path} (required): Path to the file or directory to test
\end{itemize}

\subsection{Features}
\begin{itemize}[leftmargin=*]
    \item Existence check
    \item Type identification (file, directory, symlink, device)
    \item Permission testing (read, write, execute)
    \item Detailed file information (size, modification time)
    \item Visual indicators (✓ for yes, ✗ for no)
\end{itemize}

\subsection{Usage Examples}

\textbf{Example 1: Test a file}
\begin{lstlisting}[caption={Testing a file}]
MyShell:~$ Test-File /etc/passwd
\end{lstlisting}

\textbf{Example 2: Test a directory}
\begin{lstlisting}[caption={Testing a directory}]
MyShell:~$ Test-File /home/user/Documents
\end{lstlisting}

\textbf{Example 3: Test a symlink}
\begin{lstlisting}[caption={Testing a symbolic link}]
MyShell:~$ Test-File /usr/bin/python
\end{lstlisting}

\subsection{Sample Output}
\begin{tcolorbox}[colback=gray!10,colframe=gray!50]
\begin{verbatim}
=== File/Directory Test Report ===

Target: /etc/passwd

✓ Exists
✓ Regular file

Permissions:
✓ Readable
✗ Not writable
✗ Not executable

Details:
-rw-r--r-- 1 root root 2.8K Oct 15 14:23 /etc/passwd

Size: 2845 bytes
Last Modified: 2025-10-15 14:23:45

Test complete!
\end{verbatim}
\end{tcolorbox}

\subsection{Use Cases}
\begin{enumerate}[leftmargin=*]
    \item Pre-deployment file verification
    \item Debugging permission issues
    \item Script validation before execution
    \item Security audits
\end{enumerate}

\section{Server-Health.sh - System Health Monitor}

\subsection{Purpose}
Comprehensive system health monitoring tool that provides real-time information about system uptime, load, memory usage, disk space, and process statistics.

\subsection{Syntax}
\begin{lstlisting}[frame=single,numbers=none]
Server-Health.sh
\end{lstlisting}

\subsection{Parameters}
No parameters required.

\subsection{Features}
\begin{itemize}[leftmargin=*]
    \item System uptime display
    \item Load average (1, 5, 15 minutes)
    \item Memory usage with percentage calculation
    \item Disk usage for root filesystem
    \item Top 5 CPU-consuming processes
    \item Top 5 memory-consuming processes
    \item Health status indicators (warnings for high usage)
\end{itemize}

\subsection{Usage Example}
\begin{lstlisting}[caption={Running system health check}]
MyShell:~$ Server-Health
\end{lstlisting}

\subsection{Sample Output}
\begin{tcolorbox}[colback=gray!10,colframe=gray!50]
\begin{verbatim}
╔════════════════════════════════════════════════════════╗
║         Server Health Monitoring Report               ║
╚════════════════════════════════════════════════════════╝

[System Uptime]
up 5 days, 3 hours, 42 minutes

[Load Average]
1-min, 5-min, 15-min: 0.52 0.48 0.51
CPU Cores: 4

[Memory Usage]
              total        used        free      shared
Mem:          7.7Gi       4.2Gi       1.8Gi       234Mi
Swap:         2.0Gi       0.0Gi       2.0Gi
Memory Usage: 54.5%
✓ Memory usage is healthy

[Disk Usage]
Filesystem      Size  Used Avail Use% Mounted on
/dev/sda1       100G   45G   50G  47% /
✓ Disk space is healthy

[Top 5 Processes by CPU]
USER       PID %CPU %MEM    VSZ   RSS TTY      STAT START   TIME COMMAND
root      1234  5.2  2.1 234560 45678 ?        Ssl  Oct15  12:34 process1
...

[Top 5 Processes by Memory]
USER       PID %CPU %MEM    VSZ   RSS TTY      STAT START   TIME COMMAND
user      5678  1.2  8.5 1234567 234567 ?      Sl   Oct16  23:45 process2
...

Health check complete at Mon Oct 22 15:30:45 UTC 2025
\end{verbatim}
\end{tcolorbox}

\subsection{Health Indicators}
\begin{table}[h]
\centering
\begin{tabular}{|l|l|l|}
\hline
\textbf{Metric} & \textbf{Warning Level} & \textbf{Alert Level} \\ \hline
Memory Usage & > 60\% & > 80\% \\ \hline
Disk Space & > 60\% & > 80\% \\ \hline
Load Average & > CPU Cores & > 2 × CPU Cores \\ \hline
\end{tabular}
\caption{Health Monitoring Thresholds}
\label{tab:health_thresholds}
\end{table}

\subsection{Use Cases}
\begin{enumerate}[leftmargin=*]
    \item Regular server health checks
    \item Troubleshooting performance issues
    \item Capacity planning
    \item System monitoring dashboards
    \item Pre-maintenance checks
\end{enumerate}

\section{CPU.sh - CPU Usage Monitor}

\subsection{Purpose}
Monitors CPU usage and alerts when usage exceeds a specified threshold.

\subsection{Syntax}
\begin{lstlisting}[frame=single,numbers=none]
CPU.sh [threshold_percentage]
\end{lstlisting}

\subsection{Parameters}
\begin{itemize}[leftmargin=*]
    \item \textbf{threshold\_percentage} (optional): CPU usage threshold for alerts. Default: 80\%
\end{itemize}

\subsection{Features}
\begin{itemize}[leftmargin=*]
    \item Real-time CPU usage calculation
    \item Customizable alert threshold
    \item Top 5 CPU-consuming processes when threshold exceeded
    \item CPU information display (model, cores, threads)
    \item Multiple CPU detection methods (top, mpstat, /proc/stat)
\end{itemize}

\subsection{Usage Examples}

\textbf{Example 1: Default threshold (80\%)}
\begin{lstlisting}[caption={Monitoring CPU with default threshold}]
MyShell:~$ CPU
\end{lstlisting}

\textbf{Example 2: Custom threshold (90\%)}
\begin{lstlisting}[caption={Monitoring CPU with 90\% threshold}]
MyShell:~$ CPU 90
\end{lstlisting}

\textbf{Example 3: Strict threshold (50\%)}
\begin{lstlisting}[caption={Monitoring CPU with 50\% threshold}]
MyShell:~$ CPU 50
\end{lstlisting}

\subsection{Sample Output (Below Threshold)}
\begin{tcolorbox}[colback=green!10,colframe=green!50]
\begin{verbatim}
=== CPU Usage Monitor ===

Threshold: 80%
Checking CPU usage...

Current CPU Usage: 45.2%

✓ CPU usage is within acceptable limits

CPU Information:
Model name:           Intel(R) Core(TM) i5-8250U CPU @ 1.60GHz
CPU(s):               8
Thread(s) per core:   2
Core(s) per socket:   4

Monitoring complete!
\end{verbatim}
\end{tcolorbox}

\subsection{Sample Output (Above Threshold)}
\begin{tcolorbox}[colback=red!10,colframe=red!50]
\begin{verbatim}
=== CPU Usage Monitor ===

Threshold: 80%
Checking CPU usage...

Current CPU Usage: 92.5%

⚠️  ALERT: CPU usage (92.5%) exceeds threshold (80%)!

Top 5 CPU-consuming processes:
USER       PID %CPU %MEM    VSZ   RSS TTY      STAT START   TIME COMMAND
user      1234 45.2  2.1 234560 45678 ?        R    15:20   0:45 heavy_process
user      5678 23.1  1.5 123456 34567 ?        R    15:18   0:32 another_process
...
\end{verbatim}
\end{tcolorbox}

\subsection{Use Cases}
\begin{enumerate}[leftmargin=*]
    \item Performance monitoring
    \item Identifying CPU-intensive processes
    \item Server capacity planning
    \item Automated alerting systems
    \item Troubleshooting slow systems
\end{enumerate}

\subsection{Exit Codes}
\begin{itemize}[leftmargin=*]
    \item \textbf{0}: CPU usage is below threshold
    \item \textbf{1}: CPU usage exceeds threshold
\end{itemize}

\section{Disk-Space.sh - Disk Space Monitor}

\subsection{Purpose}
Monitors disk space usage across all filesystems and alerts when usage exceeds specified thresholds.

\subsection{Syntax}
\begin{lstlisting}[frame=single,numbers=none]
Disk-Space.sh [threshold_percentage]
\end{lstlisting}

\subsection{Parameters}
\begin{itemize}[leftmargin=*]
    \item \textbf{threshold\_percentage} (optional): Disk usage threshold for alerts. Default: 80\%
\end{itemize}

\subsection{Features}
\begin{itemize}[leftmargin=*]
    \item Multi-filesystem monitoring
    \item Customizable alert thresholds
    \item Health status for each filesystem
    \item Largest directories in root when alerts triggered
    \item Inode usage reporting
    \item Excludes temporary filesystems (tmpfs, devtmpfs)
\end{itemize}

\subsection{Usage Examples}

\textbf{Example 1: Default monitoring}
\begin{lstlisting}[caption={Disk space monitoring with default threshold}]
MyShell:~$ Disk-Space
\end{lstlisting}

\textbf{Example 2: Strict threshold}
\begin{lstlisting}[caption={Disk space monitoring with 70\% threshold}]
MyShell:~$ Disk-Space 70
\end{lstlisting}

\subsection{Sample Output}
\begin{tcolorbox}[colback=gray!10,colframe=gray!50]
\begin{verbatim}
╔════════════════════════════════════════════════════════╗
║           Disk Space Monitoring Report                ║
╚════════════════════════════════════════════════════════╝

Alert Threshold: 80%

[All Filesystems]
Filesystem      Size  Used Avail Use% Mounted on
/dev/sda1       100G   45G   50G  47% /
/dev/sdb1       500G  380G  100G  79% /data
/dev/sdc1        50G   42G    5G  89% /backup

[Status Check]
✓ / is healthy at 47% (Available: 50G)
⚠  Warning: /data is at 79% (Available: 100G)
⚠️  ALERT: /backup is at 89% (Available: 5G)

[Largest Directories in /]
12G     /usr
8.5G    /var
6.2G    /home
3.1G    /opt
...

[Inode Usage]
Filesystem     Inodes  IUsed  IFree IUse% Mounted on
/dev/sda1      6553600 234567 6319033    4% /

Scan completed at Mon Oct 22 15:35:12 UTC 2025
\end{verbatim}
\end{tcolorbox}

\subsection{Use Cases}
\begin{enumerate}[leftmargin=*]
    \item Regular disk space monitoring
    \item Preventing disk full situations
    \item Capacity planning
    \item Cleanup prioritization
    \item Backup verification
\end{enumerate}

\subsection{Exit Codes}
\begin{itemize}[leftmargin=*]
    \item \textbf{0}: All filesystems below threshold
    \item \textbf{1}: One or more filesystems exceed threshold
\end{itemize}

\section{CollectNetworkInfo.sh - Network Configuration Collector}

\subsection{Purpose}
Gathers comprehensive network configuration information including interfaces, connections, routing, and DNS settings.

\subsection{Syntax}
\begin{lstlisting}[frame=single,numbers=none]
CollectNetworkInfo.sh
\end{lstlisting}

\subsection{Parameters}
No parameters required.

\subsection{Features}
\begin{itemize}[leftmargin=*]
    \item Hostname display
    \item Network interfaces with IP addresses
    \item Active network connections
    \item Routing table information
    \item DNS server configuration
    \item Default gateway
    \item Public IP address (requires internet)
    \item Network statistics
\end{itemize}

\subsection{Usage Example}
\begin{lstlisting}[caption={Collecting network information}]
MyShell:~$ CollectNetworkInfo
\end{lstlisting}

\subsection{Sample Output}
\begin{tcolorbox}[colback=gray!10,colframe=gray!50]
\begin{verbatim}
╔════════════════════════════════════════════════════════╗
║          Network Configuration Report                 ║
╚════════════════════════════════════════════════════════╝

[Hostname]
myserver.example.com

[Network Interfaces]
lo      UNKNOWN  127.0.0.1/8 ::1/128
eth0    UP       192.168.1.100/24 fe80::a00:27ff:fe4e:66a1/64
wlan0   DOWN

[Active Network Connections]
Listening Ports:
Netid  State   Recv-Q Send-Q  Local Address:Port   Peer Address:Port
tcp    LISTEN  0      128     0.0.0.0:22            0.0.0.0:*
tcp    LISTEN  0      100     127.0.0.1:25          0.0.0.0:*
tcp    LISTEN  0      128     0.0.0.0:80            0.0.0.0:*

[Routing Table]
default via 192.168.1.1 dev eth0 proto dhcp metric 100
192.168.1.0/24 dev eth0 proto kernel scope link src 192.168.1.100

[DNS Servers]
nameserver 8.8.8.8
nameserver 8.8.4.4
nameserver 192.168.1.1

[Default Gateway]
default via 192.168.1.1 dev eth0

[Public IP Address]
203.0.113.45

[Network Statistics]
eth0:  RX packets:1234567 TX packets:987654
       RX bytes:12.3 GB   TX bytes:8.7 GB

Network information collection complete!
\end{verbatim}
\end{tcolorbox}

\subsection{Use Cases}
\begin{enumerate}[leftmargin=*]
    \item Network troubleshooting
    \item Documentation and auditing
    \item Network configuration verification
    \item Pre-deployment checks
    \item Security assessments
\end{enumerate}

\subsection{Tool Compatibility}
The script automatically detects and uses available tools:
\begin{itemize}[leftmargin=*]
    \item \texttt{ip} command (preferred) or \texttt{ifconfig}
    \item \texttt{ss} command (preferred) or \texttt{netstat}
    \item \texttt{curl} for public IP detection
\end{itemize}

\section{RemoteBackup.sh - Remote Backup Utility}

\subsection{Purpose}
Creates compressed backup archives and transfers them to remote servers via SCP.

\subsection{Syntax}
\begin{lstlisting}[frame=single,numbers=none]
RemoteBackup.sh <source_path> <remote_user@host> <remote_path>
\end{lstlisting}

\subsection{Parameters}
\begin{itemize}[leftmargin=*]
    \item \textbf{source\_path} (required): Local file or directory to backup
    \item \textbf{remote\_user@host} (required): Remote server credentials and hostname
    \item \textbf{remote\_path} (required): Destination path on remote server
\end{itemize}

\subsection{Features}
\begin{itemize}[leftmargin=*]
    \item Automatic compression (tar.gz format)
    \item Timestamped backup filenames
    \item Source existence validation
    \item Archive size reporting
    \item Secure transfer via SCP
    \item Automatic cleanup of temporary files
    \item Transfer verification
\end{itemize}

\subsection{Prerequisites}
\begin{itemize}[leftmargin=*]
    \item SSH access to remote server
    \item SSH key authentication recommended
    \item Write permissions on remote destination
\end{itemize}

\subsection{Usage Examples}

\textbf{Example 1: Backup directory to remote server}
\begin{lstlisting}[caption={Backing up a directory}]
MyShell:~$ RemoteBackup ~/Documents user@192.168.1.100 /backups/
\end{lstlisting}

\textbf{Example 2: Backup specific file}
\begin{lstlisting}[caption={Backing up a file}]
MyShell:~$ RemoteBackup /etc/config.conf admin@backup.example.com /backups/config/
\end{lstlisting}

\subsection{Sample Output}
\begin{tcolorbox}[colback=gray!10,colframe=gray!50]
\begin{verbatim}
=== Remote Backup Utility ===

Source: /home/user/Documents
Destination: user@192.168.1.100:/backups/
Backup Name: backup_20251022_153045.tar.gz

Creating backup archive...
✓ Archive created successfully
Archive size: 45M

Transferring to remote server...
backup_20251022_153045.tar.gz    100%   45MB   5.6MB/s   00:08

✓ Backup transferred successfully!
Remote location: user@192.168.1.100:/backups/backup_20251022_153045.tar.gz

Backup completed at Mon Oct 22 15:30:53 UTC 2025
\end{verbatim}
\end{tcolorbox}

\subsection{Use Cases}
\begin{enumerate}[leftmargin=*]
    \item Automated backup scripts
    \item Disaster recovery preparation
    \item Regular data archiving
    \item Off-site backup creation
    \item Configuration file backups
\end{enumerate}

\subsection{Error Handling}
\begin{itemize}[leftmargin=*]
    \item Validates source existence
    \item Checks SSH connectivity
    \item Verifies remote path
    \item Handles transfer failures
    \item Cleans up temporary files on error
\end{itemize}

\subsection{Security Considerations}
\begin{tcolorbox}[colback=yellow!10,colframe=yellow!75!black,title=\textbf{Security Best Practices}]
\begin{enumerate}[leftmargin=*]
    \item Use SSH key authentication instead of passwords
    \item Restrict remote user permissions
    \item Verify backup integrity after transfer
    \item Use encrypted connections only
    \item Store backups on separate physical systems
\end{enumerate}
\end{tcolorbox}

\section{HardwareInfo.sh - Hardware Information Display}

\subsection{Purpose}
Comprehensive system hardware information display tool for inventory and documentation.

\subsection{Syntax}
\begin{lstlisting}[frame=single,numbers=none]
HardwareInfo.sh
\end{lstlisting}

\subsection{Parameters}
No parameters required.

\subsection{Features}
\begin{itemize}[leftmargin=*]
    \item System information (hostname, kernel, architecture, OS)
    \item CPU details (model, cores, threads, frequency, cache)
    \item Memory information (total, free, available)
    \item Storage device listing
    \item Network interface enumeration
    \item PCI device detection
    \item USB device listing
    \item Graphics card information
    \item BIOS/UEFI details (if available)
\end{itemize}

\subsection{Usage Example}
\begin{lstlisting}[caption={Displaying hardware information}]
MyShell:~$ HardwareInfo
\end{lstlisting}

\subsection{Sample Output}
\begin{tcolorbox}[colback=gray!10,colframe=gray!50,fontupper=\small]
\begin{verbatim}
╔════════════════════════════════════════════════════════╗
║           System Hardware Information                 ║
╚════════════════════════════════════════════════════════╝

[System Information]
Hostname: mycomputer.local
Kernel: 5.15.0-56-generic
Architecture: x86_64
OS: Ubuntu 22.04.1 LTS

[CPU Information]
Model name:           Intel(R) Core(TM) i7-8750H CPU @ 2.20GHz
Architecture:         x86_64
CPU(s):               12
Thread(s) per core:   2
Core(s) per socket:   6
CPU MHz:              2200.000
L1d cache:            32K
L1i cache:            32K
L2 cache:             256K
L3 cache:             9216K

[Memory Information]
              total        used        free      shared  buff/cache   available
Mem:            15Gi       6.2Gi       2.1Gi       456Mi       6.8Gi       8.5Gi
Swap:          2.0Gi          0B       2.0Gi

MemTotal:       16384000 kB
MemFree:        2147483 kB
MemAvailable:   8912345 kB

[Storage Devices]
NAME   SIZE TYPE MOUNTPOINT
sda    500G disk
├─sda1 100G part /
├─sda2 380G part /data
nvme0n1 256G disk
└─nvme0n1p1 256G part /home

[Network Interfaces]
1: lo: <LOOPBACK,UP,LOWER_UP>
2: eth0: <BROADCAST,MULTICAST,UP,LOWER_UP>
3: wlan0: <BROADCAST,MULTICAST>

[PCI Devices]
00:02.0 VGA compatible controller: Intel Corporation UHD Graphics 630
00:14.0 USB controller: Intel Corporation Sunrise Point-H USB 3.0 xHCI
00:1f.3 Audio device: Intel Corporation Cannon Lake PCH cAVS
01:00.0 3D controller: NVIDIA Corporation GP107M [GeForce GTX 1050]
02:00.0 Ethernet controller: Realtek Semiconductor RTL8111/8168/8411

[USB Devices]
Bus 001 Device 002: ID 8087:0024 Intel Corp. Integrated Rate Matching Hub
Bus 001 Device 003: ID 046d:c52b Logitech, Inc. Unifying Receiver
Bus 002 Device 001: ID 1d6b:0003 Linux Foundation 3.0 root hub

[Graphics Information]
00:02.0 VGA compatible controller: Intel Corporation UHD Graphics 630
01:00.0 3D controller: NVIDIA Corporation GP107M [GeForce GTX 1050]

[BIOS Information]
BIOS Vendor: American Megatrends Inc.
BIOS Version: F.32
BIOS Date: 06/12/2020

Hardware scan complete!
\end{verbatim}
\end{tcolorbox}

\subsection{Use Cases}
\begin{enumerate}[leftmargin=*]
    \item System inventory and asset management
    \item Technical documentation
    \item Hardware upgrade planning
    \item Troubleshooting and support tickets
    \item Compatibility verification
\end{enumerate}

\section{Get-Temperature.sh - CPU Temperature Monitor}

\subsection{Purpose}
Monitors CPU temperature using system sensors with threshold-based alerts.

\subsection{Syntax}
\begin{lstlisting}[frame=single,numbers=none]
Get-Temperature.sh
\end{lstlisting}

\subsection{Parameters}
No parameters required.

\subsection{Features}
\begin{itemize}[leftmargin=*]
    \item Real-time temperature readings
    \item Multiple temperature sources (sensors, thermal zones)
    \item Average temperature calculation
    \item Temperature warnings (60°C, 80°C thresholds)
    \item Installation instructions if sensors unavailable
    \item Fallback methods for different systems
\end{itemize}

\subsection{Prerequisites}
\textbf{Recommended:} \texttt{lm-sensors} package

\begin{lstlisting}[caption={Installing lm-sensors}]
# Ubuntu/Debian
sudo apt-get install lm-sensors
sudo sensors-detect

# Fedora/RHEL
sudo dnf install lm_sensors
sudo sensors-detect

# Arch Linux
sudo pacman -S lm_sensors
sudo sensors-detect
\end{lstlisting}

\subsection{Usage Example}
\begin{lstlisting}[caption={Monitoring CPU temperature}]
MyShell:~$ Get-Temperature
\end{lstlisting}

\subsection{Sample Output (With Sensors)}
\begin{tcolorbox}[colback=gray!10,colframe=gray!50]
\begin{verbatim}
=== CPU Temperature Monitor ===

[Sensor Readings]
Core 0:        +45.0°C  (high = +100.0°C, crit = +100.0°C)
Core 1:        +47.0°C  (high = +100.0°C, crit = +100.0°C)
Core 2:        +44.0°C  (high = +100.0°C, crit = +100.0°C)
Core 3:        +46.0°C  (high = +100.0°C, crit = +100.0°C)

Average Temperature: 45.5°C
✓ Temperature is normal

Temperature check complete!
\end{verbatim}
\end{tcolorbox}

\subsection{Sample Output (Without Sensors)}
\begin{tcolorbox}[colback=yellow!10,colframe=yellow!50]
\begin{verbatim}
=== CPU Temperature Monitor ===

sensors command not found, trying alternative methods...

[Thermal Zones]
thermal_zone0: 48°C
thermal_zone1: 52°C

Temperature check complete!
\end{verbatim}
\end{tcolorbox}

\subsection{Temperature Thresholds}
\begin{table}[h]
\centering
\begin{tabular}{|l|l|l|}
\hline
\textbf{Temperature} & \textbf{Status} & \textbf{Action} \\ \hline
< 60°C & Normal & ✓ Continue operation \\ \hline
60°C - 80°C & Elevated & ⚠ Monitor closely \\ \hline
> 80°C & High & ⚠️ Check cooling system \\ \hline
> 90°C & Critical & 🔥 Immediate attention \\ \hline
\end{tabular}
\caption{Temperature Warning Levels}
\label{tab:temp_thresholds}
\end{table}

\subsection{Use Cases}
\begin{enumerate}[leftmargin=*]
    \item System health monitoring
    \item Overheating prevention
    \item Cooling system verification
    \item Overclocking safety
    \item Data center monitoring
\end{enumerate}

\subsection{Troubleshooting}
\begin{tcolorbox}[colback=blue!5,colframe=blue!50,title=\textbf{Common Issues}]
\textbf{Problem:} No temperature sensors found

\textbf{Solution:}
\begin{enumerate}[leftmargin=*]
    \item Install lm-sensors: \texttt{sudo apt-get install lm-sensors}
    \item Run sensor detection: \texttt{sudo sensors-detect}
    \item Answer "YES" to all driver detection prompts
    \item Reboot or load kernel modules manually
\end{enumerate}
\end{tcolorbox}

% Chapter 5: Network Services Scripts
\chapter{Network Services Scripts}

\section{tiny-http.sh - Minimal HTTP Server}

\subsection{Purpose}
A minimal HTTP server implementation using netcat that can serve static files from a directory.

\subsection{Syntax}
\begin{lstlisting}[frame=single,numbers=none]
tiny-http.sh [port] [directory]
\end{lstlisting}

\subsection{Parameters}
\begin{itemize}[leftmargin=*]
    \item \textbf{port} (optional): Port number to listen on. Default: 8080
    \item \textbf{directory} (optional): Directory to serve files from. Default: current directory
\end{itemize}

\subsection{Features}
\begin{itemize}[leftmargin=*]
    \item Serves static files (HTML, CSS, JS, images)
    \item Auto-generates directory index
    \item Supports multiple MIME types
    \item Request logging with timestamps
    \item Proper HTTP headers
    \item Handles 404 Not Found errors
    \item Graceful shutdown (Ctrl+C)
\end{itemize}

\subsection{Usage Examples}

\textbf{Example 1: Default server on port 8080}
\begin{lstlisting}[caption={Starting HTTP server with defaults}]
MyShell:~$ tiny-http
\end{lstlisting}

\textbf{Example 2: Custom port}
\begin{lstlisting}[caption={Starting HTTP server on port 3000}]
MyShell:~$ tiny-http 3000
\end{lstlisting}

\textbf{Example 3: Custom directory and port}
\begin{lstlisting}[caption={Serving files from /var/www/html on port 80}]
MyShell:~$ sudo tiny-http 80 /var/www/html
\end{lstlisting}

\subsection{Sample Output}
\begin{tcolorbox}[colback=gray!10,colframe=gray!50]
\begin{verbatim}
╔════════════════════════════════════════════════════════╗
║              Tiny HTTP Server v1.0                    ║
╚════════════════════════════════════════════════════════╝

Server Configuration:
  Port: 8080
  Document Root: /home/user/public
  Server URL: http://localhost:8080/

Creating index.html...
✓ Auto-generated index page created

Starting server...
✓ Server is running!

Press Ctrl+C to stop the server

[2025-10-22 15:40:12] GET / - 200 OK
[2025-10-22 15:40:15] GET /style.css - 200 OK
[2025-10-22 15:40:18] GET /script.js - 200 OK
[2025-10-22 15:40:22] GET /favicon.ico - 404 Not Found

^C
Server stopped.
\end{verbatim}
\end{tcolorbox}

\subsection{Supported MIME Types}
\begin{table}[h]
\centering
\begin{tabular}{|l|l|}
\hline
\textbf{Extension} & \textbf{MIME Type} \\ \hline
.html, .htm & text/html \\ \hline
.css & text/css \\ \hline
.js & application/javascript \\ \hline
.json & application/json \\ \hline
.jpg, .jpeg & image/jpeg \\ \hline
.png & image/png \\ \hline
.gif & image/gif \\ \hline
.txt & text/plain \\ \hline
.pdf & application/pdf \\ \hline
.zip & application/zip \\ \hline
\end{tabular}
\caption{Supported File Types}
\label{tab:mime_types}
\end{table}

\subsection{Use Cases}
\begin{enumerate}[leftmargin=*]
    \item Quick file sharing on local network
    \item Testing web applications
    \item Serving static documentation
    \item Temporary development server
    \item Educational purposes (learning HTTP protocol)
\end{enumerate}

\subsection{Security Warning}
\begin{tcolorbox}[colback=red!10,colframe=red!75!black,title=\textbf{⚠️ Security Notice}]
This is a minimal server for development and testing purposes only. Do NOT use in production or expose to the internet. It lacks:
\begin{itemize}[leftmargin=*]
    \item Authentication mechanisms
    \item HTTPS/TLS encryption
    \item Request validation
    \item DoS protection
    \item Security headers
\end{itemize}
\end{tcolorbox}

% Chapter 6: Mathematical Operations Scripts
\chapter{Mathematical Operations Scripts}

\section{Addition.sh - Addition Calculator}

\subsection{Purpose}
Performs addition of two numbers with support for integers and floating-point values.

\subsection{Syntax}
\begin{lstlisting}[frame=single,numbers=none]
Addition.sh <number1> <number2>
\end{lstlisting}

\subsection{Parameters}
\begin{itemize}[leftmargin=*]
    \item \textbf{number1} (required): First number (integer or float)
    \item \textbf{number2} (required): Second number (integer or float)
\end{itemize}

\subsection{Features}
\begin{itemize}[leftmargin=*]
    \item Integer and floating-point arithmetic
    \item Input validation
    \item Error handling for non-numeric input
    \item Uses \texttt{bc} for precision
\end{itemize}

\subsection{Usage Examples}

\textbf{Example 1: Integer addition}
\begin{lstlisting}[caption={Adding integers}]
MyShell:~$ Addition 10 20
\end{lstlisting}

\textbf{Output:}
\begin{lstlisting}[frame=none,numbers=none]
=== Addition Calculator ===

10 + 20 = 30
\end{lstlisting}

\textbf{Example 2: Floating-point addition}
\begin{lstlisting}[caption={Adding floating-point numbers}]
MyShell:~$ Addition 15.5 8.25
\end{lstlisting}

\textbf{Output:}
\begin{lstlisting}[frame=none,numbers=none]
=== Addition Calculator ===

15.5 + 8.25 = 23.75
\end{lstlisting}

\textbf{Example 3: Negative numbers}
\begin{lstlisting}[caption={Adding negative numbers}]
MyShell:~$ Addition -50 75
\end{lstlisting}

\textbf{Output:}
\begin{lstlisting}[frame=none,numbers=none]
=== Addition Calculator ===

-50 + 75 = 25
\end{lstlisting}

\subsection{Error Handling}
\begin{lstlisting}[caption={Invalid input example}]
MyShell:~$ Addition abc 123
Error: 'abc' is not a valid number
Usage: Addition.sh <number1> <number2>
\end{lstlisting}

\section{Subtraction.sh - Subtraction Calculator}

\subsection{Purpose}
Performs subtraction of two numbers with support for integers and floating-point values.

\subsection{Syntax}
\begin{lstlisting}[frame=single,numbers=none]
Subtraction.sh <number1> <number2>
\end{lstlisting}

\subsection{Parameters}
\begin{itemize}[leftmargin=*]
    \item \textbf{number1} (required): Minuend (number to subtract from)
    \item \textbf{number2} (required): Subtrahend (number to subtract)
\end{itemize}

\subsection{Usage Examples}

\textbf{Example 1: Basic subtraction}
\begin{lstlisting}[caption={Subtracting numbers}]
MyShell:~$ Subtraction 50 30
\end{lstlisting}

\textbf{Output:}
\begin{lstlisting}[frame=none,numbers=none]
=== Subtraction Calculator ===

50 - 30 = 20
\end{lstlisting}

\textbf{Example 2: Floating-point subtraction}
\begin{lstlisting}[caption={Subtracting decimals}]
MyShell:~$ Subtraction 99.99 25.50
\end{lstlisting}

\textbf{Output:}
\begin{lstlisting}[frame=none,numbers=none]
=== Subtraction Calculator ===

99.99 - 25.50 = 74.49
\end{lstlisting}

\section{Multiplication.sh - Multiplication Calculator}

\subsection{Purpose}
Performs multiplication of two numbers with support for integers and floating-point values.

\subsection{Syntax}
\begin{lstlisting}[frame=single,numbers=none]
Multiplication.sh <number1> <number2>
\end{lstlisting}

\subsection{Parameters}
\begin{itemize}[leftmargin=*]
    \item \textbf{number1} (required): First factor
    \item \textbf{number2} (required): Second factor
\end{itemize}

\subsection{Usage Examples}

\textbf{Example 1: Integer multiplication}
\begin{lstlisting}[caption={Multiplying integers}]
MyShell:~$ Multiplication 12 8
\end{lstlisting}

\textbf{Output:}
\begin{lstlisting}[frame=none,numbers=none]
=== Multiplication Calculator ===

12 * 8 = 96
\end{lstlisting}

\textbf{Example 2: Decimal multiplication}
\begin{lstlisting}[caption={Multiplying decimals}]
MyShell:~$ Multiplication 3.14 2
\end{lstlisting}

\textbf{Output:}
\begin{lstlisting}[frame=none,numbers=none]
=== Multiplication Calculator ===

3.14 * 2 = 6.28
\end{lstlisting}

\section{Division.sh - Division Calculator}

\subsection{Purpose}
Performs division of two numbers with floating-point precision and division by zero protection.

\subsection{Syntax}
\begin{lstlisting}[frame=single,numbers=none]
Division.sh <number1> <number2>
\end{lstlisting}

\subsection{Parameters}
\begin{itemize}[leftmargin=*]
    \item \textbf{number1} (required): Dividend (number to be divided)
    \item \textbf{number2} (required): Divisor (number to divide by)
\end{itemize}

\subsection{Features}
\begin{itemize}[leftmargin=*]
    \item Floating-point division
    \item Division by zero detection
    \item Configurable precision (default: 2 decimal places)
\end{itemize}

\subsection{Usage Examples}

\textbf{Example 1: Integer division}
\begin{lstlisting}[caption={Dividing integers}]
MyShell:~$ Division 100 4
\end{lstlisting}

\textbf{Output:}
\begin{lstlisting}[frame=none,numbers=none]
=== Division Calculator ===

100 / 4 = 25.00
\end{lstlisting}

\textbf{Example 2: Floating-point division}
\begin{lstlisting}[caption={Dividing decimals}]
MyShell:~$ Division 22 7
\end{lstlisting}

\textbf{Output:}
\begin{lstlisting}[frame=none,numbers=none]
=== Division Calculator ===

22 / 7 = 3.14
\end{lstlisting}

\textbf{Example 3: Division by zero}
\begin{lstlisting}[caption={Division by zero error}]
MyShell:~$ Division 10 0
\end{lstlisting}

\textbf{Output:}
\begin{lstlisting}[frame=none,numbers=none]
=== Division Calculator ===

Error: Division by zero is not allowed!
\end{lstlisting}

\section{Simplecalc.sh - Interactive Calculator}

\subsection{Purpose}
An interactive calculator providing seven mathematical operations through a menu-driven interface.

\subsection{Syntax}
\begin{lstlisting}[frame=single,numbers=none]
Simplecalc.sh
\end{lstlisting}

\subsection{Parameters}
No parameters required (interactive mode).

\subsection{Features}
\begin{itemize}[leftmargin=*]
    \item Seven operations: Addition, Subtraction, Multiplication, Division, Power, Square Root, Modulo
    \item Interactive menu system
    \item Input validation
    \item Continuous calculation mode
    \item Error handling for all operations
\end{itemize}

\subsection{Available Operations}
\begin{enumerate}[leftmargin=*]
    \item Addition ($a + b$)
    \item Subtraction ($a - b$)
    \item Multiplication ($a \times b$)
    \item Division ($a \div b$)
    \item Power ($a^b$)
    \item Square Root ($\sqrt{a}$)
    \item Modulo ($a \bmod b$)
\end{enumerate}

\subsection{Usage Example}
\begin{lstlisting}[caption={Using the interactive calculator}]
MyShell:~$ Simplecalc
\end{lstlisting}

\subsection{Sample Session}
\begin{tcolorbox}[colback=gray!10,colframe=gray!50]
\begin{verbatim}
╔════════════════════════════════════════════════════════╗
║            Simple Calculator v1.0                     ║
╚════════════════════════════════════════════════════════╝

Select Operation:
  1) Addition
  2) Subtraction
  3) Multiplication
  4) Division
  5) Power
  6) Square Root
  7) Modulo
  8) Exit

Enter your choice [1-8]: 5

Enter first number: 2
Enter second number: 10

Result: 2 ^ 10 = 1024

Press Enter to continue...

Select Operation:
[... menu repeats ...]

Enter your choice [1-8]: 6

Enter number: 144

Result: √144 = 12

Press Enter to continue...

Enter your choice [1-8]: 8

Thank you for using Simple Calculator!
\end{verbatim}
\end{tcolorbox}

\subsection{Use Cases}
\begin{enumerate}[leftmargin=*]
    \item Quick calculations without leaving terminal
    \item Educational tool for learning bash scripting
    \item Scientific calculations
    \item Financial calculations
\end{enumerate}

% Chapter 7: Interactive Utilities Scripts
\chapter{Interactive Utilities Scripts}

\section{Hello.sh - Input/Output Demonstration}

\subsection{Purpose}
Demonstrates basic input/output operations in Bash, including user interaction and variable manipulation.

\subsection{Syntax}
\begin{lstlisting}[frame=single,numbers=none]
Hello.sh
\end{lstlisting}

\subsection{Parameters}
No parameters required (interactive).

\subsection{Features}
\begin{itemize}[leftmargin=*]
    \item User name input
    \item Personalized greeting
    \item Current date and time display
    \item Color-formatted output
    \item Demonstrates basic I/O concepts
\end{itemize}

\subsection{Usage Example}
\begin{lstlisting}[caption={Running the Hello script}]
MyShell:~$ Hello
\end{lstlisting}

\subsection{Sample Session}
\begin{tcolorbox}[colback=gray!10,colframe=gray!50]
\begin{verbatim}
╔════════════════════════════════════════════════════════╗
║            Hello World I/O Demo                       ║
╚════════════════════════════════════════════════════════╝

Please enter your name: Alice

Hello, Alice!
Welcome to the Custom Bash Shell Environment.

Current Information:
  Date: Monday, October 22, 2025
  Time: 15:45:30 UTC
  System: Linux 5.15.0-56-generic

Have a great day, Alice!
\end{verbatim}
\end{tcolorbox}

\subsection{Use Cases}
\begin{enumerate}[leftmargin=*]
    \item Learning Bash I/O operations
    \item Template for interactive scripts
    \item User welcome scripts
    \item Basic personalization examples
\end{enumerate}

\section{Process.sh - Command Chaining Demonstration}

\subsection{Purpose}
Demonstrates various command chaining techniques including pipes, redirects, and command substitution.

\subsection{Syntax}
\begin{lstlisting}[frame=single,numbers=none]
Process.sh
\end{lstlisting}

\subsection{Parameters}
No parameters required.

\subsection{Features}
\begin{itemize}[leftmargin=*]
    \item Pipe examples (|)
    \item Redirect examples (>, >>, <)
    \item Command substitution (\$(command))
    \item Process substitution
    \item Logical operators (\&\&, ||)
    \item Background processes (\&)
    \item Educational commentary
\end{itemize}

\subsection{Usage Example}
\begin{lstlisting}[caption={Running process demonstration}]
MyShell:~$ Process
\end{lstlisting}

\subsection{Sample Output}
\begin{tcolorbox}[colback=gray!10,colframe=gray!50,fontupper=\small]
\begin{verbatim}
╔════════════════════════════════════════════════════════╗
║        Command Chaining & Process Management          ║
╚════════════════════════════════════════════════════════╝

[1] Pipe Example (|)
Command: ls -l | grep ".sh" | wc -l
Output: 22

[2] Output Redirection (>)
Command: echo "Hello World" > /tmp/test.txt
✓ Written to file

[3] Append Redirection (>>)
Command: echo "Second line" >> /tmp/test.txt
✓ Appended to file

[4] Command Substitution
Command: echo "Today is $(date +%A)"
Output: Today is Monday

[5] Logical AND (&&)
Command: mkdir /tmp/testdir && cd /tmp/testdir && pwd
Output: /tmp/testdir
✓ All commands succeeded

[6] Logical OR (||)
Command: false || echo "Previous command failed"
Output: Previous command failed

[7] Background Process (&)
Command: sleep 5 &
Process ID: 12345
✓ Running in background

Demonstration complete!
\end{verbatim}
\end{tcolorbox}

\subsection{Learning Objectives}
\begin{enumerate}[leftmargin=*]
    \item Understanding command pipelines
    \item File redirection techniques
    \item Process control and job management
    \item Conditional command execution
    \item Command substitution patterns
\end{enumerate}

\section{Interactive.sh - Multi-Level Menu System}

\subsection{Purpose}
Demonstrates advanced interactive menu navigation with multiple levels, submenus, and user choices.

\subsection{Syntax}
\begin{lstlisting}[frame=single,numbers=none]
Interactive.sh
\end{lstlisting}

\subsection{Parameters}
No parameters required (interactive).

\subsection{Features}
\begin{itemize}[leftmargin=*]
    \item Multi-level menu hierarchy
    \item Color-coded options
    \item Input validation
    \item Breadcrumb navigation
    \item Nested submenu support
    \item Context-aware help
\end{itemize}

\subsection{Menu Structure}
\begin{enumerate}[leftmargin=*]
    \item \textbf{Main Menu}
    \begin{itemize}
        \item System Information
        \item File Operations
        \item Network Tools
        \item Settings
        \item Exit
    \end{itemize}
    \item \textbf{System Information Submenu}
    \begin{itemize}
        \item CPU Info
        \item Memory Info
        \item Disk Usage
        \item Back to Main Menu
    \end{itemize}
    \item \textbf{File Operations Submenu}
    \begin{itemize}
        \item List Files
        \item Search Files
        \item File Statistics
        \item Back to Main Menu
    \end{itemize}
    \item \textbf{Network Tools Submenu}
    \begin{itemize}
        \item Show Interfaces
        \item Check Connectivity
        \item Port Scanner
        \item Back to Main Menu
    \end{itemize}
\end{enumerate}

\subsection{Usage Example}
\begin{lstlisting}[caption={Running interactive menu}]
MyShell:~$ Interactive
\end{lstlisting}

\subsection{Sample Session}
\begin{tcolorbox}[colback=gray!10,colframe=gray!50]
\begin{verbatim}
╔════════════════════════════════════════════════════════╗
║         Interactive Menu System Demo                  ║
╚════════════════════════════════════════════════════════╝

>>> MAIN MENU <<<

1) System Information
2) File Operations
3) Network Tools
4) Settings
5) Exit

Enter your choice [1-5]: 1

>>> SYSTEM INFORMATION <<<

1) CPU Info
2) Memory Info
3) Disk Usage
4) Back to Main Menu

Enter your choice [1-4]: 2

Fetching memory information...

MemTotal:       16384000 kB
MemFree:        8192000 kB
MemAvailable:   10240000 kB

Press Enter to continue...

[Returns to System Information menu]
\end{verbatim}
\end{tcolorbox}

\subsection{Use Cases}
\begin{enumerate}[leftmargin=*]
    \item Creating user-friendly CLI applications
    \item System administration interfaces
    \item Educational demonstrations
    \item Guided troubleshooting tools
\end{enumerate}

\section{pomodoro.sh - Productivity Timer}

\subsection{Purpose}
Implements the Pomodoro Technique for time management with work/break cycles.

\subsection{Syntax}
\begin{lstlisting}[frame=single,numbers=none]
pomodoro.sh [work_minutes] [break_minutes]
\end{lstlisting}

\subsection{Parameters}
\begin{itemize}[leftmargin=*]
    \item \textbf{work\_minutes} (optional): Work session duration. Default: 25 minutes
    \item \textbf{break\_minutes} (optional): Break duration. Default: 5 minutes
\end{itemize}

\subsection{Features}
\begin{itemize}[leftmargin=*]
    \item Customizable work and break durations
    \item Progress bar visualization
    \item Audio notifications (if available)
    \item Cycle counter
    \item Pause/resume functionality
    \item Session statistics
\end{itemize}

\subsection{Usage Examples}

\textbf{Example 1: Default Pomodoro (25/5)}
\begin{lstlisting}[caption={Standard Pomodoro timer}]
MyShell:~$ pomodoro
\end{lstlisting}

\textbf{Example 2: Custom duration (45/10)}
\begin{lstlisting}[caption={Extended work session}]
MyShell:~$ pomodoro 45 10
\end{lstlisting}

\textbf{Example 3: Short sprint (15/3)}
\begin{lstlisting}[caption={Quick focus session}]
MyShell:~$ pomodoro 15 3
\end{lstlisting}

\subsection{Sample Session}
\begin{tcolorbox}[colback=gray!10,colframe=gray!50]
\begin{verbatim}
╔════════════════════════════════════════════════════════╗
║            Pomodoro Timer v1.0                        ║
╚════════════════════════════════════════════════════════╝

Configuration:
  Work Duration: 25 minutes
  Break Duration: 5 minutes

Starting Pomodoro #1...

🍅 WORK SESSION (25:00)
Progress: [████████████████████░░░░░░░░] 20:15 remaining

[After work session completes:]
✓ Work session complete! Time for a break.

☕ BREAK TIME (5:00)
Progress: [██████░░░░░░░░░░░░░░░░░░░░░░] 3:45 remaining

[After break:]
✓ Break complete! Ready for next session?
  (y/n): y

Starting Pomodoro #2...
\end{verbatim}
\end{tcolorbox}

\subsection{Pomodoro Technique}
\begin{tcolorbox}[colback=blue!5,colframe=blue!50,title=\textbf{About the Pomodoro Technique}]
The Pomodoro Technique is a time management method:
\begin{enumerate}[leftmargin=*]
    \item Work focused for 25 minutes (1 Pomodoro)
    \item Take a 5-minute break
    \item After 4 Pomodoros, take a longer break (15-30 minutes)
    \item Repeat the cycle
\end{enumerate}

Benefits:
\begin{itemize}[leftmargin=*]
    \item Improved focus and concentration
    \item Reduced mental fatigue
    \item Better time awareness
    \item Increased productivity
\end{itemize}
\end{tcolorbox}

\subsection{Keyboard Controls}
\begin{table}[h]
\centering
\begin{tabular}{|l|l|}
\hline
\textbf{Key} & \textbf{Action} \\ \hline
Ctrl+C & Stop current session \\ \hline
p & Pause timer \\ \hline
r & Resume timer \\ \hline
s & Skip to next session \\ \hline
q & Quit application \\ \hline
\end{tabular}
\caption{Pomodoro Timer Controls}
\label{tab:pomodoro_controls}
\end{table}

\subsection{Use Cases}
\begin{enumerate}[leftmargin=*]
    \item Focused coding sessions
    \item Study time management
    \item Writing and creative work
    \item Task completion tracking
    \item Burnout prevention
\end{enumerate}

% Chapter 8: Miscellaneous Scripts
\chapter{Miscellaneous Scripts}

\section{weather.sh - Weather Information Display}

\subsection{Purpose}
Fetches and displays current weather information for any city using wttr.in API.

\subsection{Syntax}
\begin{lstlisting}[frame=single,numbers=none]
weather.sh [city_name]
\end{lstlisting}

\subsection{Parameters}
\begin{itemize}[leftmargin=*]
    \item \textbf{city\_name} (optional): City name for weather lookup. Default: auto-detect based on IP
\end{itemize}

\subsection{Features}
\begin{itemize}[leftmargin=*]
    \item Current weather conditions
    \item Temperature (Celsius and Fahrenheit)
    \item Weather description
    \item Wind speed and direction
    \item Humidity and visibility
    \item 3-day forecast
    \item ASCII art weather icons
    \item Auto-location detection
\end{itemize}

\subsection{Prerequisites}
\textbf{Required:} Internet connection and \texttt{curl}

\subsection{Usage Examples}

\textbf{Example 1: Auto-detect location}
\begin{lstlisting}[caption={Weather for current location}]
MyShell:~$ weather
\end{lstlisting}

\textbf{Example 2: Specific city}
\begin{lstlisting}[caption={Weather for New York}]
MyShell:~$ weather "New York"
\end{lstlisting}

\textbf{Example 3: International city}
\begin{lstlisting}[caption={Weather for Tokyo}]
MyShell:~$ weather Tokyo
\end{lstlisting}

\subsection{Sample Output}
\begin{tcolorbox}[colback=gray!10,colframe=gray!50,fontupper=\small]
\begin{verbatim}
Weather report: New York

     \  /       Partly cloudy
   _ /"".-.     22°C
     \_(   ).   ↓ 15 km/h
     /(___(__)  10 km
                0.0 mm

┌──────────────────────────────┬───────────────────────┐
│            Monday            │        Tuesday        │
├──────────────────────────────┼───────────────────────┤
│     \   /     Sunny          │    \  /   Partly      │
│      .-.      24 °C          │  _ /"".-. cloudy      │
│   ― (   ) ―   ↑ 12 km/h      │    \_(  ). 20 °C      │
│      `-'      10 km          │    /(___(__)↗ 18 km/h │
│     /   \     0.0 mm | 0%    │           10 km       │
└──────────────────────────────┴───────────────────────┘

Location: New York, New York, United States
Timezone: America/New_York
\end{verbatim}
\end{tcolorbox}

\subsection{Use Cases}
\begin{enumerate}[leftmargin=*]
    \item Quick weather checks from terminal
    \item Travel planning
    \item Script integration for weather-dependent tasks
    \item System information dashboards
\end{enumerate}

\subsection{API Information}
\begin{itemize}[leftmargin=*]
    \item \textbf{Service:} wttr.in
    \item \textbf{Rate Limit:} None (fair use expected)
    \item \textbf{Data Source:} Multiple weather services
    \item \textbf{Documentation:} https://github.com/chubin/wttr.in
\end{itemize}

\section{RedditTop.sh - Reddit Top Posts Viewer}

\subsection{Purpose}
Fetches and displays top posts from any subreddit using Reddit's JSON API.

\subsection{Syntax}
\begin{lstlisting}[frame=single,numbers=none]
RedditTop.sh [subreddit] [limit]
\end{lstlisting}

\subsection{Parameters}
\begin{itemize}[leftmargin=*]
    \item \textbf{subreddit} (optional): Subreddit name. Default: "linux"
    \item \textbf{limit} (optional): Number of posts to display (1-25). Default: 10
\end{itemize}

\subsection{Features}
\begin{itemize}[leftmargin=*]
    \item Top posts from any subreddit
    \item Customizable post count
    \item Post title, score, author, and comment count
    \item Direct link to each post
    \item Color-coded output
    \item JSON parsing (jq preferred, falls back to grep/sed)
\end{itemize}

\subsection{Prerequisites}
\begin{itemize}[leftmargin=*]
    \item \textbf{Required:} Internet connection, \texttt{curl}
    \item \textbf{Recommended:} \texttt{jq} for better formatting
\end{itemize}

\subsection{Usage Examples}

\textbf{Example 1: Default (r/linux, 10 posts)}
\begin{lstlisting}[caption={Viewing top Linux posts}]
MyShell:~$ RedditTop
\end{lstlisting}

\textbf{Example 2: Custom subreddit}
\begin{lstlisting}[caption={Viewing top programming posts}]
MyShell:~$ RedditTop programming
\end{lstlisting}

\textbf{Example 3: Custom subreddit and limit}
\begin{lstlisting}[caption={Viewing top 5 Python posts}]
MyShell:~$ RedditTop python 5
\end{lstlisting}

\subsection{Sample Output}
\begin{tcolorbox}[colback=gray!10,colframe=gray!50,fontupper=\small]
\begin{verbatim}
╔════════════════════════════════════════════════════════╗
║          Reddit Top Posts: r/linux                    ║
╚════════════════════════════════════════════════════════╝

[1] ⬆ 2.4k | 💬 156 | u/linuxuser123
    Title: Linux kernel 6.6 released with amazing features
    https://www.reddit.com/r/linux/comments/abc123/

[2] ⬆ 1.8k | 💬 89 | u/opensource_fan
    Title: New distro release: AwesomeLinux 2.0
    https://www.reddit.com/r/linux/comments/def456/

[3] ⬆ 1.5k | 💬 234 | u/terminal_wizard
    Title: My minimal terminal setup after 10 years
    https://www.reddit.com/r/linux/comments/ghi789/

[4] ⬆ 1.2k | 💬 67 | u/bash_master
    Title: Bash scripting tips that changed my workflow
    https://www.reddit.com/r/linux/comments/jkl012/

[5] ⬆ 980 | 💬 145 | u/kernel_dev
    Title: Understanding Linux filesystem hierarchy
    https://www.reddit.com/r/linux/comments/mno345/

[6-10 continued...]

Fetched at: Mon Oct 22 16:00:15 UTC 2025
\end{verbatim}
\end{tcolorbox}

\subsection{Use Cases}
\begin{enumerate}[leftmargin=*]
    \item Quick Reddit browsing from terminal
    \item Staying updated on favorite subreddits
    \item Research and content discovery
    \item Integration with notification systems
\end{enumerate}

\subsection{Rate Limiting}
\begin{tcolorbox}[colback=yellow!10,colframe=yellow!75!black,title=\textbf{API Usage Note}]
Reddit's API has rate limits:
\begin{itemize}[leftmargin=*]
    \item Unauthenticated requests: 60/hour
    \item This script uses public JSON endpoint
    \item Respect rate limits for fair usage
    \item Consider caching for frequent use
\end{itemize}
\end{tcolorbox}

\section{Colorful.sh - Terminal Color Demonstration}

\subsection{Purpose}
Demonstrates terminal color capabilities including 16-color, 256-color, and RGB support.

\subsection{Syntax}
\begin{lstlisting}[frame=single,numbers=none]
Colorful.sh
\end{lstlisting}

\subsection{Parameters}
No parameters required.

\subsection{Features}
\begin{itemize}[leftmargin=*]
    \item 16 basic ANSI colors
    \item 256-color palette display
    \item RGB true color examples
    \item Text formatting (bold, italic, underline, etc.)
    \item Background colors
    \item Color combination examples
    \item Terminal capability detection
\end{itemize}

\subsection{Usage Example}
\begin{lstlisting}[caption={Running color demonstration}]
MyShell:~$ Colorful
\end{lstlisting}

\subsection{Sample Output}
\begin{tcolorbox}[colback=gray!10,colframe=gray!50,fontupper=\small]
\begin{verbatim}
╔════════════════════════════════════════════════════════╗
║        Terminal Color Demonstration                   ║
╚════════════════════════════════════════════════════════╝

[16 Basic Colors]
  Black   Red   Green   Yellow   Blue   Magenta   Cyan   White

[Text Formatting]
  Normal Text
  Bold Text
  Italic Text
  Underlined Text
  Strikethrough Text

[256-Color Palette]
  [Displays color grid with numbers 0-255]

[RGB True Color Examples]
  [Displays gradient bars and color examples]

[Background Colors]
  [Shows text with various background colors]

[Fancy Examples]
  [Displays rainbow text, gradients, and combinations]

Your terminal supports: 256 colors + True color (24-bit)
\end{verbatim}
\end{tcolorbox}

\subsection{Color Codes Reference}
\begin{table}[h]
\centering
\small
\begin{tabular}{|l|l|l|}
\hline
\textbf{Format} & \textbf{Code} & \textbf{Example} \\ \hline
Foreground & \texttt{\textbackslash 033[30-37m} & \texttt{\textbackslash 033[31m} (Red) \\ \hline
Background & \texttt{\textbackslash 033[40-47m} & \texttt{\textbackslash 033[44m} (Blue BG) \\ \hline
Bright FG & \texttt{\textbackslash 033[90-97m} & \texttt{\textbackslash 033[92m} (Bright Green) \\ \hline
256-color FG & \texttt{\textbackslash 033[38;5;Nm} & \texttt{\textbackslash 033[38;5;202m} \\ \hline
RGB FG & \texttt{\textbackslash 033[38;2;R;G;Bm} & \texttt{\textbackslash 033[38;2;255;100;0m} \\ \hline
Bold & \texttt{\textbackslash 033[1m} & Bold text \\ \hline
Reset & \texttt{\textbackslash 033[0m} & Reset all \\ \hline
\end{tabular}
\caption{ANSI Color Code Reference}
\label{tab:color_codes}
\end{table}

\subsection{Use Cases}
\begin{enumerate}[leftmargin=*]
    \item Learning terminal color codes
    \item Testing terminal capabilities
    \item Designing colored CLI applications
    \item Creating visually appealing scripts
    \item Terminal theme development
\end{enumerate}

\subsection{Terminal Compatibility}
\begin{itemize}[leftmargin=*]
    \item \textbf{Full Support:} Modern terminals (GNOME Terminal, Konsole, iTerm2, Windows Terminal)
    \item \textbf{Partial Support:} Older terminals (limited to 16 or 256 colors)
    \item \textbf{Minimal Support:} Very old terminals (monochrome only)
\end{itemize}

% Chapter 9: Advanced Topics
\chapter{Advanced Topics}

\section{Extending the Shell}

\subsection{Adding Custom Scripts}

To add your own scripts to the shell environment:

\begin{enumerate}[leftmargin=*]
    \item Create your script in the \texttt{scripts/} directory
    \item Make it executable: \texttt{chmod +x scripts/your\_script.sh}
    \item Follow the naming convention: use .sh extension
    \item Add appropriate help text and error handling
    \item The shell will automatically discover it
\end{enumerate}

\textbf{Example Script Template:}
\begin{lstlisting}[caption={Template for new scripts}]
#!/bin/bash

# Script: MyCustomScript.sh
# Purpose: Brief description
# Author: Your Name
# Date: 2025-10-22

# Color definitions
RED='\033[0;31m'
GREEN='\033[0;32m'
BLUE='\033[0;34m'
NC='\033[0m' # No Color

# Function definitions
function show_usage() {
    echo "Usage: $0 <parameters>"
    echo "Description: What this script does"
}

function main() {
    # Parameter validation
    if [ $# -lt 1 ]; then
        show_usage
        exit 1
    fi
    
    # Main logic here
    echo -e "${GREEN}Script executing...${NC}"
    
    # Your code here
    
    echo -e "${GREEN}Complete!${NC}"
}

# Execute main function
main "$@"
\end{lstlisting}

\subsection{Customizing the Shell}

The shell can be customized by editing \texttt{myshell.sh}:

\begin{itemize}[leftmargin=*]
    \item \textbf{Colors:} Modify color definitions at the top
    \item \textbf{Prompt:} Change the \texttt{PS1} variable
    \item \textbf{Welcome Banner:} Edit \texttt{show\_banner()} function
    \item \textbf{History Size:} Adjust \texttt{HISTSIZE} variable
    \item \textbf{Script Directory:} Change \texttt{SCRIPTS\_DIR} path
\end{itemize}

\section{Automation and Scheduling}

\subsection{Cron Integration}

Many scripts can be scheduled using cron:

\begin{lstlisting}[caption={Example crontab entries}]
# Daily disk space check at 8 AM
0 8 * * * /path/to/shell/scripts/Disk-Space.sh 80 | mail -s "Disk Space Report" admin@example.com

# Hourly CPU monitoring
0 * * * * /path/to/shell/scripts/CPU.sh 90

# Weekly backup every Sunday at 2 AM
0 2 * * 0 /path/to/shell/scripts/RemoteBackup.sh /data user@backup:/backups/

# Server health check every 6 hours
0 */6 * * * /path/to/shell/scripts/Server-Health.sh > /var/log/health.log
\end{lstlisting}

\subsection{Systemd Service}

Create a systemd service for monitoring:

\begin{lstlisting}[caption={Example systemd service file}]
[Unit]
Description=CPU Monitoring Service
After=network.target

[Service]
Type=oneshot
ExecStart=/path/to/shell/scripts/CPU.sh 85
User=monitor
StandardOutput=journal

[Install]
WantedBy=multi-user.target
\end{lstlisting}

\section{Integration with Other Tools}

\subsection{Logging Integration}

Send output to syslog:

\begin{lstlisting}[caption={Logging to syslog}]
# From within a script
logger -t "myshell" "CPU usage exceeded threshold"

# Pipe script output to logger
./scripts/Server-Health.sh | logger -t "health-check"
\end{lstlisting}

\subsection{Notification Systems}

Integrate with desktop notifications:

\begin{lstlisting}[caption={Desktop notification example}]
# Ubuntu/GNOME
if [ $(CPU.sh 90) -eq 1 ]; then
    notify-send "High CPU Alert" "CPU usage is above 90%"
fi

# macOS
if [ $(CPU.sh 90) -eq 1 ]; then
    osascript -e 'display notification "CPU above 90%" with title "Alert"'
fi
\end{lstlisting}

\subsection{Email Alerts}

Configure email notifications:

\begin{lstlisting}[caption={Email alert configuration}]
# Using mailutils
./scripts/Disk-Space.sh 90 | mail -s "Disk Space Alert" admin@example.com

# Using mutt
./scripts/Server-Health.sh | mutt -s "Health Report" -a /tmp/report.txt -- admin@example.com
\end{lstlisting}

\section{Performance Optimization}

\subsection{Script Optimization Tips}

\begin{enumerate}[leftmargin=*]
    \item Use built-in commands instead of external programs when possible
    \item Minimize subshell creation
    \item Use arrays for batch operations
    \item Cache frequently accessed data
    \item Avoid unnecessary loops
\end{enumerate}

\textbf{Example Optimization:}
\begin{lstlisting}[caption={Before and after optimization}]
# Inefficient (multiple subshells)
for file in $(ls *.txt); do
    cat $file | grep "pattern"
done

# Efficient (fewer forks)
grep "pattern" *.txt
\end{lstlisting}

\subsection{Parallel Execution}

Use GNU Parallel for batch operations:

\begin{lstlisting}[caption={Parallel script execution}]
# Sequential
for server in server1 server2 server3; do
    ./scripts/RemoteBackup.sh /data user@$server:/backup/
done

# Parallel (requires GNU parallel)
parallel ./scripts/RemoteBackup.sh /data user@{}:/backup/ ::: server1 server2 server3
\end{lstlisting}

% Appendices
\appendix

\chapter{Troubleshooting Guide}

\section{Common Issues and Solutions}

\subsection{Permission Denied Errors}

\textbf{Problem:} Cannot execute scripts

\textbf{Solution:}
\begin{lstlisting}[caption={Fixing permissions}]
# Make all scripts executable
chmod +x scripts/*.sh
chmod +x myshell.sh

# Check current permissions
ls -l scripts/
\end{lstlisting}

\subsection{Command Not Found}

\textbf{Problem:} Scripts not found when running from shell

\textbf{Solution:}
\begin{enumerate}[leftmargin=*]
    \item Verify scripts directory: \texttt{ls scripts/}
    \item Check \texttt{SCRIPTS\_DIR} variable in myshell.sh
    \item Ensure scripts have .sh extension
    \item Run \texttt{list} command to see available scripts
\end{enumerate}

\subsection{Dependency Issues}

\textbf{Problem:} "command not found" errors in scripts

\textbf{Solution:}
\begin{lstlisting}[caption={Installing missing dependencies}]
# Ubuntu/Debian
sudo apt-get install curl netcat bc jq lm-sensors

# Fedora/RHEL
sudo dnf install curl nc bc jq lm_sensors

# Arch Linux
sudo pacman -S curl gnu-netcat bc jq lm_sensors
\end{lstlisting}

\subsection{Network Scripts Failing}

\textbf{Problem:} weather.sh or RedditTop.sh not working

\textbf{Solutions:}
\begin{enumerate}[leftmargin=*]
    \item Check internet connectivity: \texttt{ping 8.8.8.8}
    \item Verify curl is installed: \texttt{which curl}
    \item Test API directly: \texttt{curl wttr.in}
    \item Check firewall settings
    \item Verify proxy configuration if behind corporate firewall
\end{enumerate}

\subsection{Temperature Sensors Not Detected}

\textbf{Problem:} Get-Temperature.sh shows no sensors

\textbf{Solution:}
\begin{lstlisting}[caption={Setting up lm-sensors}]
# Install sensors
sudo apt-get install lm-sensors

# Detect sensors
sudo sensors-detect
# Answer YES to all prompts

# Load modules (or reboot)
sudo service kmod start

# Test
sensors
\end{lstlisting}

\section{Debugging Techniques}

\subsection{Enable Debug Mode}

\begin{lstlisting}[caption={Running scripts in debug mode}]
# Enable bash debug output
bash -x ./scripts/DirectorySize.sh /home

# Add to script temporarily
#!/bin/bash -x

# Or within script
set -x  # Enable debug
# ... code to debug ...
set +x  # Disable debug
\end{lstlisting}

\subsection{Verbose Output}

\begin{lstlisting}[caption={Adding verbose logging}]
# Add to beginning of script
VERBOSE=true

# Use throughout script
if [ "$VERBOSE" = true ]; then
    echo "DEBUG: Variable value is $var"
fi
\end{lstlisting}

\subsection{Error Handling}

\begin{lstlisting}[caption={Comprehensive error handling}]
#!/bin/bash

# Exit on error
set -e

# Exit on undefined variable
set -u

# Exit on pipe failure
set -o pipefail

# Error handler function
trap 'echo "Error on line $LINENO"' ERR
\end{lstlisting}

\chapter{Script Reference Tables}

\section{Complete Script List}

\begin{longtable}{|p{4cm}|p{3cm}|p{8cm}|}
\hline
\textbf{Script Name} & \textbf{Category} & \textbf{Purpose} \\ \hline
\endhead

DirectorySize.sh & System Admin & Analyze directory sizes \\ \hline
Test-File.sh & System Admin & Test file/directory properties \\ \hline
Server-Health.sh & System Admin & Comprehensive system health check \\ \hline
CPU.sh & System Admin & Monitor CPU usage with alerts \\ \hline
Disk-Space.sh & System Admin & Monitor disk space usage \\ \hline
CollectNetworkInfo.sh & System Admin & Gather network configuration \\ \hline
RemoteBackup.sh & System Admin & Remote backup via SCP \\ \hline
HardwareInfo.sh & System Admin & Display hardware information \\ \hline
Get-Temperature.sh & System Admin & Monitor CPU temperature \\ \hline
tiny-http.sh & Network & Minimal HTTP server \\ \hline
Addition.sh & Math & Add two numbers \\ \hline
Subtraction.sh & Math & Subtract two numbers \\ \hline
Multiplication.sh & Math & Multiply two numbers \\ \hline
Division.sh & Math & Divide two numbers \\ \hline
Simplecalc.sh & Math & Interactive calculator \\ \hline
Hello.sh & Interactive & I/O demonstration \\ \hline
Process.sh & Interactive & Command chaining examples \\ \hline
Interactive.sh & Interactive & Multi-level menu system \\ \hline
pomodoro.sh & Interactive & Productivity timer \\ \hline
weather.sh & Miscellaneous & Weather information \\ \hline
RedditTop.sh & Miscellaneous & Reddit top posts viewer \\ \hline
Colorful.sh & Miscellaneous & Terminal color demo \\ \hline

\caption{Complete Script Reference}
\label{tab:all_scripts}
\end{longtable}

\section{Quick Command Reference}

\begin{table}[h]
\centering
\begin{tabular}{|l|p{10cm}|}
\hline
\textbf{Command} & \textbf{Description} \\ \hline
help & Show help information \\ \hline
list & List all available scripts \\ \hline
cd [dir] & Change directory \\ \hline
pwd & Print working directory \\ \hline
clear & Clear screen \\ \hline
history & Show command history \\ \hline
exit / quit & Exit the shell \\ \hline
\end{tabular}
\caption{Built-in Commands}
\label{tab:builtin_commands}
\end{table}

\chapter{Glossary}

\begin{description}[leftmargin=3cm,style=nextline]
    \item[Bash] Bourne Again Shell, a Unix shell and command language
    \item[Built-in Command] Command implemented directly in the shell
    \item[CLI] Command Line Interface
    \item[Cron] Time-based job scheduler in Unix-like systems
    \item[MIME Type] Media type identifier for file formats
    \item[Pipe] Mechanism for inter-process communication (|)
    \item[REPL] Read-Eval-Print Loop, interactive programming environment
    \item[Script] Executable file containing shell commands
    \item[Shell] Command-line interpreter for operating systems
    \item[Subshell] Child shell process spawned by parent shell
    \item[Systemd] System and service manager for Linux
\end{description}

\chapter{Additional Resources}

\section{Online Resources}

\begin{itemize}[leftmargin=*]
    \item \textbf{Bash Manual:} \url{https://www.gnu.org/software/bash/manual/}
    \item \textbf{Advanced Bash Scripting Guide:} \url{https://tldp.org/LDP/abs/html/}
    \item \textbf{ShellCheck:} \url{https://www.shellcheck.net/} (Shell script analyzer)
    \item \textbf{Bash Hackers Wiki:} \url{https://wiki.bash-hackers.org/}
    \item \textbf{Linux Command Library:} \url{https://man7.org/linux/man-pages/}
\end{itemize}

\section{Recommended Books}

\begin{enumerate}[leftmargin=*]
    \item \textit{Learning the bash Shell} by Cameron Newham
    \item \textit{Bash Cookbook} by Carl Albing and JP Vossen
    \item \textit{The Linux Command Line} by William Shotts
    \item \textit{Unix Power Tools} by Shelley Powers et al.
    \item \textit{Classic Shell Scripting} by Arnold Robbins and Nelson H.F. Beebe
\end{enumerate}

\section{Community and Support}

\begin{itemize}[leftmargin=*]
    \item \textbf{Stack Overflow:} Questions tagged with [bash] and [shell]
    \item \textbf{Reddit:} r/bash, r/commandline, r/linux
    \item \textbf{IRC:} \#bash on irc.libera.chat
    \item \textbf{Mailing Lists:} bug-bash@gnu.org
\end{itemize}

\chapter{License Information}

\section{MIT License}

\begin{tcolorbox}[colback=gray!10,colframe=gray!50,fontupper=\small]
\begin{verbatim}
MIT License

Copyright (c) 2025 Custom Bash Shell Environment

Permission is hereby granted, free of charge, to any person obtaining a copy
of this software and associated documentation files (the "Software"), to deal
in the Software without restriction, including without limitation the rights
to use, copy, modify, merge, publish, distribute, sublicense, and/or sell
copies of the Software, and to permit persons to whom the Software is
furnished to do so, subject to the following conditions:

The above copyright notice and this permission notice shall be included in all
copies or substantial portions of the Software.

THE SOFTWARE IS PROVIDED "AS IS", WITHOUT WARRANTY OF ANY KIND, EXPRESS OR
IMPLIED, INCLUDING BUT NOT LIMITED TO THE WARRANTIES OF MERCHANTABILITY,
FITNESS FOR A PARTICULAR PURPOSE AND NONINFRINGEMENT. IN NO EVENT SHALL THE
AUTHORS OR COPYRIGHT HOLDERS BE LIABLE FOR ANY CLAIM, DAMAGES OR OTHER
LIABILITY, WHETHER IN AN ACTION OF CONTRACT, TORT OR OTHERWISE, ARISING FROM,
OUT OF OR IN CONNECTION WITH THE SOFTWARE OR THE USE OR OTHER DEALINGS IN THE
SOFTWARE.
\end{verbatim}
\end{tcolorbox}

\section{Third-Party Components}

This project uses the following external services:
\begin{itemize}[leftmargin=*]
    \item \textbf{wttr.in:} Weather data (Igor Chubin) - Free to use
    \item \textbf{Reddit JSON API:} Public data access
    \item \textbf{GNU Core Utilities:} GPL licensed
    \item \textbf{bc:} GPL licensed
\end{itemize}

\chapter{Index}

\textit{Note: In a complete LaTeX document, you would use the \textbackslash index and \textbackslash printindex commands to generate an automatic index. For brevity, key terms are listed below.}

\begin{multicols}{2}
\begin{itemize}[leftmargin=*]
    \item Addition.sh
    \item API integration
    \item Automation
    \item Background processes
    \item Bash scripting
    \item Built-in commands
    \item Colorful.sh
    \item Command chaining
    \item CPU monitoring
    \item Cron integration
    \item Custom shell
    \item Debugging
    \item DirectorySize.sh
    \item Disk monitoring
    \item Division.sh
    \item Error handling
    \item Get-Temperature.sh
    \item HardwareInfo.sh
    \item Hello.sh
    \item HTTP server
    \item Input validation
    \item Installation
    \item Interactive.sh
    \item Logging
    \item Menu system
    \item MIME types
    \item Multiplication.sh
    \item Network services
    \item Performance optimization
    \item Permissions
    \item Pomodoro technique
    \item Process.sh
    \item Reddit integration
    \item RedditTop.sh
    \item RemoteBackup.sh
    \item Script categories
    \item Server-Health.sh
    \item Simplecalc.sh
    \item Subtraction.sh
    \item System administration
    \item System requirements
    \item Temperature monitoring
    \item Test-File.sh
    \item tiny-http.sh
    \item Troubleshooting
    \item weather.sh
\end{itemize}
\end{multicols}

% Back matter
\cleardoublepage
\phantomsection
\addcontentsline{toc}{chapter}{About the Author}
\chapter*{About the Author}

This comprehensive Custom Bash Shell Environment was developed as a professional demonstration of advanced shell scripting techniques, system administration automation, and software engineering best practices. The project showcases expertise in:

\begin{itemize}[leftmargin=*]
    \item Linux system programming and administration
    \item Bash scripting and automation
    \item Network programming and APIs
    \item Software architecture and design patterns
    \item Technical documentation
    \item Open-source development
\end{itemize}

The project is released under the MIT License and welcomes contributions from the community.

\vfill

\begin{center}
\textit{Thank you for using the Custom Bash Shell Environment!}

\vspace{1cm}

For updates, issues, and contributions, visit:\\
\textbf{GitHub:} \url{https://github.com/yourusername/custom-bash-shell}

\vspace{1cm}

\textbf{Happy Scripting!}
\end{center}

\end{document}
